\documentclass[DM,authoryear,toc]{lsstdoc}
% lsstdoc documentation: https://lsst-texmf.lsst.io/lsstdoc.html
\input{meta}

% Package imports go here.

% Local commands go here.

%If you want glossaries
%\input{aglossary.tex}
%\makeglossaries

\title{Charge to the Rubin Operations Survey Evaluation Working Group}

% Optional subtitle
% \setDocSubtitle{A subtitle}

\author{%
Leanne Guy
}

\setDocRef{RTN-007}
\setDocUpstreamLocation{\url{https://github.com/rubin-observatory/rtn-007}}

\date{\vcsDate}

% Optional: name of the document's curator
% \setDocCurator{The Curator of this Document}

\setDocAbstract{%
This document provides the charge to the Rubin Observatory Legacy Survey of Space and Time (LSST) Survey Evaluation Working Group (SEWG),  a Rubin Operations internal group responsible for evaluating the current and expected performance of the survey and scheduler.}

% Change history defined here.
% Order: oldest first.
% Fields: VERSION, DATE, DESCRIPTION, OWNER NAME.
% See LPM-51 for version number policy.
\setDocChangeRecord{%
  \addtohist{0.1}{2020-09-29}{Initial draft.}{Leanne Guy}
}

\begin{document}

% Create the title page.
\maketitle
% Frequently for a technote we do not want a title page  uncomment this to remove the title page and changelog.
% use \mkshorttitle to remove the extra pages

% ADD CONTENT HERE
% You can also use the \input command to include several content files.

%%%%% Introduction
\section{Introduction}
The Survey Scheduling team in the System Performance department is responsible for ensuring that the observing strategy is on track to achieve the survey's 10-yr science goals by making changes to the observing strategy. 
The Survey Evaluation Working Group (SEWG), a Rubin Operations internal group led by the System Performance department and including representation from the Observatory Operations and Data Production departments, will  support the Survey Scheduling team to this end. 

 %%%%% Scope
\section{Scope}
The SEWG is charged with evaluating the current and expected performance of the survey, as presented to them by the Survey Scheduling team, and making recommendations to the Director on any needed tactical changes to the survey cadence (e.g., due to the impact of weather, telescope or data processing responses to the schedule, etc),  to ensure the survey remains on track to achieve its 10-year science goals.  

%%%%% Responsibilities 
\section{Responsibilities}
The SWEG has the following responsibilities: 
\begin{itemize}
\item Evaluate the survey and scheduler performance.
\item Produce survey and scheduler performance reports on a quarterly basis.
\item Report to the Survey Cadence Optimization Committee (SCOC).
\item Ingest feedback from the SCOC.
\item Recommend tactical changes in the scheduling to the director of operations. 
\end{itemize}

 
%%%%% Specific Tasks
\section{Specific Tasks}

\subsection{Evaluate the survey and scheduler performance}
The SEWG will evaluate quarterly the current and expected performance of the survey and scheduler based on analysis from the Survey Scheduling Team.
Achieving this goal will involve simulating the remaining survey time; folding in an evolving understanding of the Observatory system and ascertaining whether a change to the Scheduling algorithm or configuration may be warranted. 

\subsection{Produce quarterly performance reports}
Produce quarterly reports detailing the current and expected performance of the survey and scheduler for presentation to the director of Rubin operations and the SCOC. 

\subsection{Report to the Survey Cadence Optimization Committee (SCOC)}
The SEWG will present the expected performance of the survey and scheduler to the SCOC,  a community-based sub-committee of the Science Advisory Committee (SAC). 
The SCOC will convene twice a year to evaluate the performance of the survey with respect to the science priorities of the community and to advise the Director on strategic decisions related to survey design and overall survey cadence (e.g., due to  yearly weather trends, changes to the community's science priorities, etc).
The SEWG will ingest feedback from the SCOC.


\subsection{Recommend tactical changes in the scheduling}
Based on the  survey and scheduler performance, the SEWG will recommend tactical changes in the scheduling to the director of Rubin operations to ensure the survey remains on track to achieve the 10-yr science goals.
In addition to meeting basic survey requirements, the SEWG should consider how to maximize the breadth of science that can be done with LSST by making minor changes to the observing strategy. 


%%%%% Roles 
\section{Roles and Membership}

The SEWG is led by the System Performance department and includes representation from the Observatory Operations and Data Production departments. 
Given that the SEWG will endure  throughout the full 10-year survey,  the specific membership may change however the roles are not expected change. 
For this reason, we list here the roles to be fulfilled together with the current incumbent. 

\begin{itemize}
\item {
Lead Scheduling Scientist (chair)  
\newline{Incumbent:} Lynne Jones
}
\item{
Lead Observing Specialist
\newline{Incumbent:} 
}
\item{
Observatory Support Scientist
\newline{Incumbent:}
}
\item{
Algorithms and Pipelines Lead (or delegate)
\newline{Incumbent:} 
}
\item{
Data Release Pipeline Lead (or delegate)
\newline{Incumbent:} 
}
\item{
Alert Production Lead (or delgate)
\newline{Incumbent:} 
}
\end{itemize}

% Reporting 
\section{Schedule and Reporting}

The SEWG will be convened starting FY21 as a part of the pre-operations activity and will continue through the duration of the full 10-year survey.  
During the pre-operations period, the proto-SEWG may not necessarily meet or report on the same cadence as    during the operations era. 
The SEWG chair shall report directly to the AD for System Performance and the Director for Operations. 


\appendix
% Include all the relevant bib files.
% https://lsst-texmf.lsst.io/lsstdoc.html#bibliographies
\section{References} \label{sec:bib}
\renewcommand{\refname}{} % Suppress default Bibliography section
\bibliography{local,lsst,lsst-dm,refs_ads,refs,books}

% Make sure lsst-texmf/bin/generateAcronyms.py is in your path
\section{Acronyms} \label{sec:acronyms}
\addtocounter{table}{-1}
\begin{longtable}{p{0.145\textwidth}p{0.8\textwidth}}\hline
\textbf{Acronym} & \textbf{Description}  \\\hline

 &  \\\hline
AD & Associate Director \\\hline
EFD & Engineering and Facility Database \\\hline
FY21 & Financial Year 21 \\\hline
LSST & Legacy Survey of Space and Time (formerly Large Synoptic Survey Telescope) \\\hline
OPS & Operations \\\hline
RTN & Rubin Technical Note \\\hline
SAC & Science Advisory Committee \\\hline
SCOC & Survey Cadence Optimization Committee \\\hline
SEWG & Survey Evaluation Working Group \\\hline
\end{longtable}

% If you want glossary uncomment below -- comment out the two lines above
%\printglossaries


\end{document}
